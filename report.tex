% !TEX TS-program = pdflatex
% !TEX encoding = UTF-8 Unicode

% This is a simple template for a LaTeX document using the "article" class.
% See "book", "report", "letter" for other types of document.

\documentclass[11pt]{article} % use larger type; default would be 10pt

\usepackage[utf8]{inputenc} % set input encoding (not needed with XeLaTeX)

%%% Examples of Article customizations
% These packages are optional, depending whether you want the features they provide.
% See the LaTeX Companion or other references for full information.

%%% PAGE DIMENSIONS
\usepackage{geometry} % to change the page dimensions
\geometry{a4paper} % or letterpaper (US) or a5paper or....
% \geometry{margin=2in} % for example, change the margins to 2 inches all round
% \geometry{landscape} % set up the page for landscape
%   read geometry.pdf for detailed page layout information

\usepackage{graphicx} % support the \includegraphics command and options

% \usepackage[parfill]{parskip} % Activate to begin paragraphs with an empty line rather than an indent

%%% PACKAGES
\usepackage{booktabs} % for much better looking tables
\usepackage{array} % for better arrays (eg matrices) in maths
\usepackage{paralist} % very flexible & customisable lists (eg. enumerate/itemize, etc.)
\usepackage{verbatim} % adds environment for commenting out blocks of text & for better verbatim
\usepackage{subfig} % make it possible to include more than one captioned figure/table in a single float
% These packages are all incorporated in the memoir class to one degree or another...
\usepackage{float}

%%% HEADERS & FOOTERS
\usepackage{fancyhdr} % This should be set AFTER setting up the page geometry
\pagestyle{fancy} % options: empty , plain , fancy
\renewcommand{\headrulewidth}{0pt} % customise the layout...
\lhead{}\chead{}\rhead{}
\lfoot{}\cfoot{\thepage}\rfoot{}

%%% SECTION TITLE APPEARANCE
\usepackage{sectsty}
\allsectionsfont{\sffamily\mdseries\upshape} % (See the fntguide.pdf for font help)
% (This matches ConTeXt defaults)

%%% ToC (table of contents) APPEARANCE
\usepackage[nottoc,notlof,notlot]{tocbibind} % Put the bibliography in the ToC
\usepackage[titles,subfigure]{tocloft} % Alter the style of the Table of Contents
\renewcommand{\cftsecfont}{\rmfamily\mdseries\upshape}
\renewcommand{\cftsecpagefont}{\rmfamily\mdseries\upshape} % No bold!

%%% END Article customizations

%%% The "real" document content comes below...

\title{Controlling an Animatronic Chimpanzee Head}
\author{JP Sheehan}
%\date{} % Activate to display a given date or no date (if empty),
         % otherwise the current date is printed 

\begin{document}
\maketitle

\section{Introduction}

The Alive Chimpanzee is a toy designed by WowWee Alive, a division of toy-maker WowWee Limited. %% cite https://en.wikipedia.org/wiki/WowWee_Alive_Chimpanzee
It includes eight motors and nine sensors that allow it to mimic a real chimpanzee.
The original product included an embedded system that would allow the Chimpanzee to be operated by remote control or autonomously via its sensors.

%% Include image

A modified version of the Alive Chimpanzee has been donated to the University of Canterbury.
This modified chimpanzee has had its original control board removed.
Instead, the sensors and motor control lines are wired to a set of pin headers.
The goal of this project is to create an embedded system to allow the Alive Chimpanzee's motors to be controlled via software running on another system via some connection.
An optional goal of this project is to interface with the other sensors in the system.

\section{Background}

\subsection{Inspection}

The base of the product contains the power switch, battery housings, and power socket.
It was apparent that the product runs on a 6 V DC power supply, either by four D-size batteries, or a centre-positive 6 V (3.5 A) barrel-jack connector.
The FCC application for the radio transmitter included the user manual. %% cite FCC application
This provided information about the general operation of the device, including the location and purpose of the internal sensors and motors.

The motors provide a high level of control over the chimpanzee's facial expressions.
Each motor is a standard DC type and includes a feedback signal that is used to determine the angle of the actuated part.
The following functions are provided by the motors:
\begin{itemize}
	\item Head yaw control.
	\item Head pitch control.
	\item Jaw control.
	\item Upper lip control.
	\item Shared eyebrow control.
	\item Shared eyelid control.
	\item Shared eye direction control (requires seperate motors for horizontal and vertical directions).
\end{itemize}

Along with the motor feedback lines, the following sensors are included:
\begin{itemize}
	\item Two infra-red sensors (one in each nostril).
	\item Chin touch sensor.
	\item Front-of-head touch sensor.
	\item Back-of-head touch sensor.
	\item Two microphones (one in each ear).
	\item Two touch sensors (one in each ear).
\end{itemize}

There is also a speaker located in the neck of the product.

\section{Method}

A development board was opted to be used instead of a bare microcontroller.
This was done to speed up the time taken to prototype a solution.

The type of microcontroller used in this project is constrained by the number of inputs and outputs (table~\ref{tab:gpio}).
For meeting the primary goal of this project, 8 ADC inputs and 8 ??? outputs are required.
To meet these requirements, the Arduino XYZ was chosen as the development board.

\begin{table}[h]
	\label{tab:gpio}
	\caption{The required inputs and outputs (relative to the microcontroller).}
	\centering
	\begin{tabular}{ |c|c|c| }
		\hline
		\textbf{Signal} & \textbf{Type} & \textbf{Direction} \\
		\hline
		Head yaw motor feedback & Analogue & Input \\
		Head pitch motor feedback & Analogue & Input \\
		Jaw motor feedback & Analogue & Input \\
		Upper lip motor feedback & Analogue & Input \\
		Eyebrow motor feedback & Analogue & Input \\
		Eyelid motor feedback & Analogue & Input \\
		Horizontal eye motor feedback & Analogue & Input \\
		Vertical eye motor feedback & Analogue & Input \\
		Head yaw motor control & ??? & Output \\
		Head pitch motor control & ??? & Output \\
		Jaw motor control & ??? & Output \\
		Upper lip motor control & ??? & Output \\
		Eyebrow motor control & ??? & Output \\
		Eyelid motor control & ??? & Output \\
		Horizontal eye motor control & ??? & Output \\
		Vertical eye motor control & ??? & Output \\
		\hline
	\end{tabular}
\end{table}

\section{Results}

\section{Discussion}

\section{Conclusion}


\end{document}
